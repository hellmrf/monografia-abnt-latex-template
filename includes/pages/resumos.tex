\setlength{\absparsep}{18pt} % ajusta o espaçamento dos parágrafos do resumo

\begin{resumo}
    Este documento constitui uma tese fictícia elaborada com o objetivo de demonstrar a aplicação do \LaTeX{} na redação de trabalhos acadêmicos segundo as normas da Associação Brasileira de Normas Técnicas (ABNT), utilizando a classe abn\TeX 2. Ao longo dos capítulos, são apresentados exemplos comentados de comandos, ambientes e estruturas comumente utilizados em dissertações e teses, com ênfase na conformidade tipográfica e na padronização exigida por instituições de ensino e pesquisa. Este trabalho destina-se a estudantes, docentes e demais interessados que desejam dominar o uso do \LaTeX\ no contexto acadêmico brasileiro.

    \textbf{Palavras-chave:} LaTeX. ABNT. abnTeX2. Modelo de tese. Formatação acadêmica.
\end{resumo}


% resumo em inglês
\begin{resumo}[Abstract]
    \begin{otherlanguage*}{english}
        This document is a fictional thesis developed to demonstrate the use of \LaTeX{} for writing academic documents according to the standards of the Brazilian Association of Technical Standards (ABNT), through the abn\TeX 2 class. Throughout its chapters, it presents annotated examples of commonly used commands, environments, and structures in dissertations and theses, with emphasis on typographic quality and the standardization required by academic and research institutions. This work is intended for students, faculty, and others interested in mastering \LaTeX\ within the Brazilian academic context.

        \textbf{Keywords:} LaTeX. ABNT. abnTeX2. Thesis template. Academic formatting.
    \end{otherlanguage*}
\end{resumo}
