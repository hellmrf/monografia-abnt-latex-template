\usepackage[utf8]{inputenc}      % Permite digitar acentos e caracteres especiais diretamente
\usepackage[LGR,T1]{fontenc}     % Carrega as codificações de fonte latina (T1) e grega (LGR)

\usepackage{indentfirst}
\usepackage[dvipsnames]{xcolor}
\usepackage{graphicx}
\usepackage{microtype}
\usepackage{lipsum}
\usepackage{csquotes}

% Fonte do documento (Latin Modern)
\usepackage{lmodern}

% Meus pacotes
\usepackage{amsmath}
\usepackage{amssymb}
\usepackage{siunitx}
\usepackage{tabularray}

% ----------------
% Bibliografia
\usepackage[backend=biber, style=abnt]{biblatex}
\addbibresource[glob]{bib/*.bib}


% ----------------
% Metadados

\titulo{Template de projeto \LaTeX\ para monografia: baseado em abn\TeX 2}
\autor{Heliton Martins Reis Filho}
\local{Belo Horizonte}
\data{2025}
\orientador{Prof. Dr. João da Silva}
\coorientador{Equipe \abnTeX}
\instituicao{%
Universidade Federal de Minas Gerais

Instituto de Ciências Exatas

Departamento de Química
}

\tipotrabalho{Tese (doutorado)}

\preambulo{Tese de doutorado apresentada ao Departamento de Química da Universidade Federal de Minas Gerais como requisito parcial para a obtenção de grau de Doutor em Química.}


% ----------------
% Comandos personalizados

\newcommand*{\RR}{\mathbb{R}}


% ----------------
% Configurações de pacotes

% siunitx
\sisetup{detect-weight=true, detect-family=true, separate-uncertainty=true, output-decimal-marker={,}}
\DeclareSIUnit\angstrom{\text {Å}}

% tabularray
\UseTblrLibrary{booktabs}
\UseTblrLibrary{siunitx}

% Cores no PDF
\definecolor{blue}{RGB}{41,5,195}
\definecolor{darkblue}{RGB}{28,3,136}

% PDF (hyperref)
\makeatletter
\hypersetup{
    %pagebackref=true,
    pdftitle={\@title},
    pdfauthor={\@author},
    pdfsubject={\imprimirpreambulo},
    pdfcreator={LaTeX with abnTeX2},
    pdfkeywords={trabalho de conclusão de curso}{TCC},
    colorlinks=true,       % false: boxed links; true: colored links
    linkcolor=blue,        % color of internal links
    citecolor=blue,        % color of links to bibliography
    filecolor=magenta,     % color of file links
    urlcolor=blue,
    bookmarksdepth=4
}
\makeatother

% ----------------
% Espaçamentos entre linhas e parágrafos

% Indentação do parágrafo
\setlength{\parindent}{1.3cm}

% Espaçamento entre um parágrafo e outro
\setlength{\parskip}{0.2cm}  % tente também \onelineskip


% ----------------

% Posiciona figuras e tabelas no topo da página quando adicionadas sozinhas
% em um página em branco. Ver https://github.com/abntex/abntex2/issues/170
\makeatletter
\setlength{\@fptop}{5pt} % Set distance from top of page to first float
\makeatother

% ----------------
% Índice
\makeindex
