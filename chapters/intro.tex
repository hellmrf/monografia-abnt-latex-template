\chapter{Introdução} \label{chap:intro}

A excelência tipográfica é um elemento essencial da comunicação científica. Contudo, a formatação de documentos acadêmicos representa, frequentemente, um grande obstáculo para o autor, um desafio especialmente notório em áreas como Matemática, Física e Química. A dependência de equações, fórmulas e representações complexas torna a preparação desses trabalhos em sistemas de escrita tradicionais uma tarefa árdua e ineficiente.

Este modelo foi concebido como um guia prático para a elaboração de trabalhos acadêmicos de alta qualidade com \LaTeX, garantindo total conformidade com as normas da ABNT. O documento está estruturado para ser explorado de forma interativa: para cada elemento visual, há um trecho de código-fonte correspondente que demonstra sua implementação.

A discussão começa na \autoref{sec:historico}, que apresenta um breve histórico e a motivação por trás dos modernos sistemas de composição de texto. Posteriormente, a \autoref{sec:latex-abnt} foca nas especificidades da aplicação das normas ABNT no ecossistema \LaTeX. Recomenda-se, portanto, a leitura deste guia em paralelo com a consulta ao seu código-fonte. Esperamos que este material seja mais do que um template funcional: que ele sirva como inspiração para a arte e a técnica da escrita científica.

\section{Breve histórico: \TeX, \LaTeX\ e além} \label{sec:historico}

Donald E. Knuth, professor emérito de Stanford, é uma das figuras mais influentes da ciência da computação teórica. Sua obra seminal, a série de livros \emph{The Art of Computer Programming} \cite{Knuth-TAOCP}, é um marco da literatura técnica, e suas contribuições foram reconhecidas com o Prêmio ACM Turing, frequentemente chamado de ``Prêmio Nobel da Computação'' \cite{KnuthTuring1974}.

A gênese do \TeX\ remonta à década de 1970 e nasceu de uma frustração pessoal de Knuth. Ele constatou que a qualidade tipográfica da reedição do segundo volume de sua obra magna \cite{Knuth-TAOCP-2} era drasticamente inferior à do primeiro \cite{Knuth-TAOCP-1}, um problema que ele descreve em detalhes:

\begin{citacao}
    Eu também havia preparado a segunda edição do Volume 2, o que exigia compor tipograficamente todo aquele livro novamente. Meus editores descobriram que, em 1976, era muito caro produzir um livro da mesma forma como havia sido feito em 1969. Além disso, o estilo tipográfico utilizado nos livros originais não estava disponível nas máquinas de fotocomposição óptica. Voei da Califórnia para Massachusetts para uma reunião de emergência. Os editores concordaram que a tipografia de qualidade era de extrema importância e, nos meses seguintes, empenharam-se em conseguir novas fontes que se aproximassem das anteriores.

    Mas os resultados foram bastante decepcionantes. [\ldots] Essas já estavam muito melhores que a primeira tentativa, mas ainda assim inaceitáveis. O ``N'' em ``NOAM'' estava inclinado; o ``ff'' em ``effect'' era escuro demais; as letras ``ip'' em ``multiple'' estavam muito próximas uma da outra; e assim por diante.

    Eu não sabia o que fazer. Havia passado 15 anos escrevendo aqueles livros, mas, se eles fossem ficar visualmente horríveis, eu não queria escrever mais nada. Como poderia me orgulhar de um produto assim?
    \cite[tradução própria]{Knuth1996Kyoto}.
\end{citacao}

Essa insatisfação foi o catalisador para uma decisão notável: Knuth interrompeu sua pesquisa em algoritmos para se dedicar a um novo desafio --- o desenvolvimento de um sistema de composição tipográfica digital. Seu objetivo era criar uma ferramenta capaz de produzir textos com precisão matemática e excelência visual profissional. Em suas palavras:

\begin{citacao}
    Eu não consegui resistir à tentação de enfrentar pessoalmente aquele problema tipográfico. Abandonei tudo o que estava fazendo --- eu havia acabado de escrever as primeiras 100 páginas do Volume 4 --- e decidi escrever programas de computador que gerassem os padrões de 0s e 1s de que meus editores e eu precisávamos para a nova edição do Volume 2
    \cite[tradução própria]{Knuth1996Kyoto}.
\end{citacao}

Desse esforço nasceu o sistema \TeX. A pronúncia correta é ``téc'', como na palavra ``técnica'', pois sua etimologia remete diretamente à palavra grega \textgreek{τέχνη}, que significa `arte' e `técnica'. O próprio Knuth explica que o nome \TeX\ é simplesmente a forma maiúscula da raiz grega \textgreek{τεχ}:

\begin{citacao}[english]
English words like ``technology'' stem from a Greek root beginning with the letters \textgreek{τεχ}\ldots; and this same Greek word means art as well as technology. Hence the name \TeX, which is an uppercase form of \textgreek{τεχ} \cite{Knuth1984}.
\end{citacao}

O \TeX{} revolucionou a editoração científica ao oferecer um controle inédito sobre a formatação, especialmente em documentos com notações matemáticas complexas. Contudo, sua curva de aprendizado era íngreme; a necessidade de interagir com comandos de baixo nível tornava o sistema desafiador para o usuário comum, abrindo caminho para a próxima evolução em sua história.

Para tornar o \TeX{} mais acessível e contornar sua inerente complexidade, Leslie Lamport propôs, em 1994, o \LaTeX\ --- um conjunto de macros que abstrai grande parte dos comandos de baixo nível do sistema original \cite{Lamport1994}. A filosofia de Lamport era introduzir uma separação clara entre conteúdo e formatação. Isso permitiria ao autor focar exclusivamente na estrutura lógica do documento --- como seções, listas, equações e citações ---, delegando ao sistema a responsabilidade pelos detalhes da composição tipográfica. Essa abordagem, que garante consistência e alta qualidade visual, estabeleceu o grande diferencial do \LaTeX{} em relação aos editores WYSIWYG%
\footnote{WYSIWYG é sigla para \emph{\foreignlanguage{english}{What You See Is What You Get}}, algo como ``o que você vê é o que obtém'' em português. Refere-se a \emph{softwares} de edição visual de texto, a exemplo de processadores como o Microsoft Word ou Google Docs.}.

O \LaTeX{} consolidou-se rapidamente como o padrão de fato na redação de documentos científicos, com ampla adoção por universidades, periódicos e conferências internacionais. Sua arquitetura modular foi um catalisador para o surgimento de um vasto ecossistema de pacotes especializados, que expandiram consideravelmente suas funcionalidades originais. A força dessa comunidade ativa de usuários e desenvolvedores é, segundo \textcite{Mittelbach2004}, o principal pilar que transformou o \LaTeX{} em uma plataforma robusta e adaptável às mais diversas áreas do conhecimento.

Atualmente, a evolução do projeto é coordenada pelo \textit{\LaTeX\ Project Team}, um grupo de voluntários que supervisiona o núcleo do sistema e sua compatibilidade com tecnologias modernas. O grupo, cujo logotipo pode ser visto na \autoref{fig:latex-project-logo}\footnote{Sugerimos observar, no código-fonte, os comandos utilizados para referenciar figuras.}, é também responsável pela padronização de pacotes essenciais e pela incorporação de melhorias solicitadas pela comunidade, garantindo a estabilidade e a longevidade do ecossistema \LaTeX.

O impacto do trabalho de Lamport foi reconhecido para além da comunidade tipográfica. Em 2013, ele recebeu o Prêmio Turing por suas contribuições fundamentais à ciência da computação, que incluem não apenas o \LaTeX, mas também seus estudos sobre lógica temporal e verificação formal de sistemas concorrentes \cite{LamportTuring2013}.

\begin{figure}
    \centering
    \includegraphics[width=0.5\textwidth]{img/LaTeX_project_logo.pdf}
    \caption{Logotipo do ``The \LaTeX\ Project'', iniciativa oficial, originalmente criada por Leslie Lamport, responsável pelo desenvolvimento, manutenção e evolução do sistema \LaTeX. Fonte: \textcite{LaTeXProjectLogo}.}
    \label{fig:latex-project-logo}
\end{figure}

\section{\LaTeX{} + ABNT} \label{sec:latex-abnt}

Um dos pilares conceituais do \LaTeX{} é o princípio da Separação de Conceitos \cite{Dijkstra1982}, que permite dissociar a escrita do conteúdo da formatação do documento. Na prática, isso significa que o autor pode se concentrar inteiramente no texto, delegando as complexas tarefas de diagramação para o sistema. Essa abordagem modular permite que a formatação seja tratada em um momento distinto --- ou, até mesmo, por especialistas diferentes.

No ecossistema \LaTeX{}, essa filosofia se materializa, em primeiro lugar, nas \emph{classes de documento}. Uma classe é um arquivo que define toda a estrutura e o estilo visual de um tipo específico de publicação, como um artigo (\texttt{article}), um relatório (\texttt{report}) ou uma tese. Ela funciona como uma camada de abstração que aplica um design consistente ao texto, liberando o autor da necessidade de gerenciar cada detalhe tipográfico individualmente.

De forma análoga, o gerenciamento da bibliografia é tratado por um conjunto de ferramentas independentes. Entre os sistemas mais modernos e poderosos está o Bib\LaTeX, que oferece controle e flexibilidade sobre a apresentação das referências e o estilo das citações no texto. Ao separar os dados bibliográficos brutos da sua formatação final, o Bib\LaTeX\ automatiza a geração das citações e da lista de referências, garantindo conformidade com o estilo bibliográfico escolhido pelo usuário.

No contexto da produção acadêmica brasileira, essa modularidade é especialmente valiosa. As normas da ABNT são notórias por suas especificidades, sendo frequentemente mais rígidas e idiossincráticas que os padrões internacionais. Tais exigências impõem um desafio técnico considerável, desviando o foco do pesquisador do conteúdo de seu trabalho para a complexa tarefa de adequação às normas.

Felizmente, a comunidade \LaTeX\ oferece soluções robustas e elegantes para esse problema. A classe \texttt{abn\TeX2} \cite{abntex2classe}, na qual este modelo se baseia, implementa de forma estrita as normas da ABNT para a estrutura de documentos acadêmicos. Em paralelo, o estilo \texttt{biblatex-abnt} \cite{biblatex-abnt} encarrega-se de formatar as citações e a bibliografia de acordo com as mesmas exigências.

Este documento, portanto, utiliza extensivamente a sinergia entre essas duas ferramentas. Ele pode ser adotado não apenas como um modelo, mas como um ponto de partida consolidado para a produção de trabalhos acadêmicos em total conformidade com os padrões brasileiros.
