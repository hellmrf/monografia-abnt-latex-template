\chapter{Introdução} \label{chap:intro}

A qualidade tipográfica de documentos acadêmicos é um elemento essencial na comunicação científica, mas frequentemente são um grande obstáculo para o autor, especialmente em áreas como Matemática, Física e Química, que dependem extensamente de equações, fórmulas e representações, que são trabalhosos de preparar nos sistemas tradicionais de escrita.

Neste modelo de trabalho, abordaremos a elaboração de trabalhos acadêmicos de alta qualidade tipográfica e em total acordo com as normas ABNT a partir do \LaTeX. Inúmeros recursos utilizados aqui podem ser observados no código-fonte deste documento.

Na \autoref{sec:historico} a seguir, abordaremos um breve histórico e motivação para esse tipo de sistemas. Na \autoref{sec:latex-abnt}, abordaremos as especificidades da utilização do \LaTeX\ seguindo as normas ABNT.

Recomendo enfaticamente que a leitura deste documento seja feita em paralelo com a consulta ao código-fonte, para que o leitor possa compreender como cada elemento foi implementado. Vários recursos avançados são utilizados aqui, e muitos deles podem ser adaptados para o seu próprio trabalho. Além de um ponto de partida para o seu próprio projeto, tome este documento como uma inspiração para a arte de escrever em \LaTeX.

\section{Breve histórico: \TeX, \LaTeX\ e além} \label{sec:historico}

O professor emérito de Stanford, Dr. Donald E. Knuth, é amplamente reconhecido como um dos mais influentes nomes da ciência da computação teórica. Autor da série seminal de livros \emph{The Art of Computer Programming} \cite{Knuth-TAOCP}, considerada um marco na literatura técnica e científica, ele também foi agraciado com o Prêmio ACM Turing \cite{KnuthTuring1974}, frequentemente comparado ao ``Prêmio Nobel da Computação''.

Na década de 1970, insatisfeito ao verificar que a qualidade tipográfica obtida no Volume 2 de seu \emph{The Art Of Computer Programming} \cite{Knuth-TAOCP-2} era consideravelmente inferior àquela obtida no anterior Volume 1 \cite{Knuth-TAOCP-1}, ele se viu sem saídas:

\begin{citacao}
    Eu também havia preparado a segunda edição do Volume 2, o que exigia compor tipograficamente todo aquele livro novamente. Meus editores descobriram que, em 1976, era muito caro produzir um livro da mesma forma como havia sido feito em 1969. Além disso, o estilo tipográfico utilizado nos livros originais não estava disponível nas máquinas de fotocomposição óptica. Voei da Califórnia para Massachusetts para uma reunião de emergência. Os editores concordaram que a tipografia de qualidade era de extrema importância e, nos meses seguintes, empenharam-se em conseguir novas fontes que se aproximassem das anteriores.

    Mas os resultados foram bastante decepcionantes. [\ldots] Essas já estavam muito melhores que a primeira tentativa, mas ainda assim inaceitáveis. O ``N'' em ``NOAM'' estava inclinado; o ``ff'' em ``effect'' era escuro demais; as letras ``ip'' em ``multiple'' estavam muito próximas uma da outra; e assim por diante.

    Eu não sabia o que fazer. Havia passado 15 anos escrevendo aqueles livros, mas, se eles fossem ficar visualmente horríveis, eu não queria escrever mais nada. Como poderia me orgulhar de um produto assim?
    \cite[tradução própria]{Knuth1996Kyoto}.
\end{citacao}

Diante dessa frustração, ele decidiu interromper temporariamente sua pesquisa em algoritmos e estruturas de dados para desenvolver um sistema de composição tipográfica capaz de produzir textos tecnicamente precisos com qualidade tipográfica profissional, efetivamente transformando sua decepção em motivação para assumir, ele próprio, aquele desafio técnico:

\begin{citacao}
    Eu não consegui resistir à tentação de enfrentar pessoalmente aquele problema tipográfico. Abandonei tudo o que estava fazendo --- eu havia acabado de escrever as primeiras 100 páginas do Volume 4 --- e decidi escrever programas de computador que gerassem os padrões de 0s e 1s de que meus editores e eu precisávamos para a nova edição do Volume 2
    \cite[tradução própria]{Knuth1996Kyoto}.
\end{citacao}

Ali surgiria o sistema de processamento \TeX, cuja pronúncia é ``tec'', como em ``tecnologia''; ou, no inglês, ``tech'' como em ``technology''. Isso porque o radical ``tecno'' (ou ``techno'', em inglês) tem raízes na palavra grega \foreignlanguage{greek}{τέχνη}, que significa arte e gera palavras como técnica (\foreignlanguage{greek}{τεχνική}) e tecnologia (\foreignlanguage{greek}{τεχνολογία}). Portanto, para \textcite{Knuth1984}, o nome \TeX\ é uma ``versão em maiúsculas'' de \foreignlanguage{greek}{τεχ}:

\begin{citacao}[english]
English words like ``technology'' stem from a Greek root beginning with the letters \foreignlanguage{greek}{τεχ}\ldots; and this same Greek word means art as well as technology. Hence the name \TeX, which is an uppercase form of \foreignlanguage{greek}{τεχ} \cite{Knuth1984}.
\end{citacao}

O \TeX{} revolucionou a composição de textos científicos ao permitir um controle preciso sobre a formatação, especialmente para documentos com notações matemáticas complexas. No entanto, sua complexidade e a necessidade de lidar diretamente com comandos de baixo nível tornavam seu uso desafiador para muitos usuários.

Para contornar esse problema e tornar o \TeX\ mais acessível, \textcite{Lamport1994} propôs o \LaTeX, um conjunto de macros que abstrai grande parte da complexidade do sistema original. A proposta de Lamport era oferecer uma separação clara entre conteúdo e formatação, permitindo que o autor se concentrasse na estrutura lógica do documento --- como seções, listas, equações e citações --- enquanto o sistema cuidava automaticamente dos aspectos tipográficos. Essa filosofia foi um dos grandes diferenciais do \LaTeX{} em relação a editores WYSIWYG%
\footnote{WYSIWYG é sigla para \emph{\foreignlanguage{english}{What You See Is What You Get}}, algo como ``o que você vê é o que obtém'' em português. Refere-se a \emph{softwares} de edição visual de texto, a exemplo de processadores como o Microsoft Word ou Google Docs.}%
, além de garantir consistência e alta qualidade tipográfica.

O \LaTeX\ rapidamente se consolidou como o padrão de fato para a redação de documentos científicos, sendo amplamente adotado por universidades, periódicos acadêmicos e conferências internacionais. Sua arquitetura modular possibilitou o desenvolvimento de pacotes especializados, que expandiram consideravelmente suas funcionalidades.

Para \textcite{Mittelbach2004}, a grande força do \LaTeX\ reside justamente nessa comunidade ativa de usuários e desenvolvedores, que ao longo das décadas contribuiu para tornar o sistema uma plataforma robusta e adaptável às mais diversas áreas do conhecimento.

Atualmente, o desenvolvimento e a manutenção do projeto é coordenado pelo \textit{\LaTeX\ Project Team}, um grupo de voluntários experientes que supervisiona tanto a evolução do núcleo do \LaTeX\ quanto a compatibilidade com outros sistemas modernos. Esse grupo, cujo logotipo pode ser visto na \autoref{fig:latex-project-logo}\footnote{Sugerimos observar, no código-fonte, os comandos utilizados para referenciar figuras.}, também é responsável pela padronização de pacotes essenciais, pela documentação oficial e pela incorporação de melhorias solicitadas pela comunidade, garantindo a estabilidade e a continuidade do ecossistema \LaTeX\ em longo prazo.

Em 2013, Leslie Lamport também recebeu o Prêmio Turing por suas contribuições fundamentais à ciência da computação, incluindo, além do \LaTeX, seus trabalhos sobre lógica temporal e verificação formal de sistemas concorrentes \cite{LamportTuring2013}.

\begin{figure}
    \centering
    \includegraphics[width=0.5\textwidth]{img/LaTeX_project_logo.pdf}
    \caption{Logotipo do ``The \LaTeX\ Project'', iniciativa oficial, originalmente criada por Leslie Lamport, responsável pelo desenvolvimento, manutenção e evolução do sistema \LaTeX. Fonte: \textcite{LaTeXProjectLogo}.}
    \label{fig:latex-project-logo}
\end{figure}

\section{\LaTeX\ + ABNT} \label{sec:latex-abnt}

O \LaTeX\ emerge como um sistema de preparação de documentos de alta qualidade tipográfica que, ao mesmo tempo, adota o princípio de Separação de Conceitos \cite{Dijkstra1982}, permitindo que a escrita do conteúdo e a formatação do documento sejam feitas em momentos distintos --- ou, até mesmo, por equipes distintas. Isso significa que o autor deve se concentrar no conteúdo, sem se preocupar com a formatação.

No universo do \LaTeX{}, \emph{classes de documento} são arquivos que definem a estrutura e a formatação geral de um tipo específico de texto, como um artigo (\verb|article|), um relatório (\verb|report|), uma apresentação (\verb|beamer|), ou uma tese. Em essência, uma classe funciona como um molde que guia a forma final do texto, facilitando a aplicação de um estilo consistente sem que o autor precise se preocupar com cada detalhe tipográfico.

Enquanto as classes lidam com a estrutura e o visual do documento como um todo, a bibliografia é tratada por outro conjunto de ferramentas. Um dos sistemas mais poderosos e modernos para gerenciamento de referências no \LaTeX{} é o Bib\LaTeX, que oferece muito mais controle e flexibilidade sobre como as referências são apresentadas e citadas no texto.
Ao separar os dados bibliográficos e a formatação desses dados, o Bib\LaTeX\ permite ao autor fazer citações sem se preocupar com o estilo de citação. As citações e bibliografia são então geradas automaticamente, de acordo com o estilo escolhido pelo usuário.

Sobre a produção acadêmica no Brasil, observaremos que as normas ABNT são muito específicas, frequentemente mais rígidas e idiossincráticas que os estilos internacionais. Isso impõe desafios técnicos para quem deseja focar no conteúdo do seu próprio trabalho, em vez de se especializar em tais normas.

Felizmente, o ecossistema \LaTeX\ oferece soluções práticas, elegantes e de altíssima qualidade para resolver esse problema. A classe abn\TeX2 \cite{abntex2classe}, na qual este documento se baseia, implementa estritamente todas as normas ABNT relacionadas à formatação de documentos acadêmicos, enquanto o estilo \verb|biblatex-abnt| \cite{biblatex-abnt} fornece citações e bibliografia conforme as normas.

Este documento utiliza extensivamente ambos, e pode ser utilizado como ponto de partida para o seu próprio trabalho.
