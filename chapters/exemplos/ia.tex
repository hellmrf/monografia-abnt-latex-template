\section{Inteligência Artificial} \label{sec:ia}

A Inteligência Artificial é um campo da ciência da computação que se dedica a desenvolver sistemas capazes de executar tarefas que normalmente exigiriam inteligência humana. Desde a sua origem, a IA tem se caracterizado por uma série de marcos teóricos e práticos que, juntos, vêm moldando a forma como o ser humano interage com a tecnologia e com o mundo moderno.

Embora a popularização do termo seja recente e esteja frequentemente associada ao surgimento da \emph{Inteligência Artificial Generativa}, marcado pelo lançamento de modelos de linguagem como o \emph{ChatGPT}, os primeiros estudos nesta área na verdade remontam à segunda metade do século XX.

Em 1950, Alan Turing publica o seu artigo intitulado \citetitle{turing1950}, no qual se concentra na definição e no estudo da ``inteligência''. Nesse trabalho, o amplamente considerado ``pai da computação'' propõe uma questão que, até o momento, não escaparia da esfera filosófica: \textbf{``As máquinas podem pensar?''} \cite{turing1950}.

Para responder a essa difícil e até então inexplorada pergunta, \citeauthor{turing1950} propõe um experimento que ficaria conhecido como o ``Teste de Turing''. A ideia é simples: um juiz humano interage com duas entidades, uma máquina e um ser humano, sem saber qual é qual. Se o juiz não for capaz de distinguir entre as respostas da máquina e do humano, então a máquina pode ser considerada ``inteligente''.

Esse trabalho não apenas inaugurou uma nova forma de pensar sobre a interação homem--máquina, mas também lançou as bases para a formalização do estudo dessa nova área.

Pouco mais tarde, ainda na década de 1950, o termo ``inteligência artificial'' é então cunhado pelo Prof. John McCarthy, hoje Professor Emérito de Ciência da Computação da Universidade Stanford, quem responde ao questionamento \emph{``o que é inteligência artificial?''} com as seguintes palavras:

\begin{citacao}[english]
   É a ciência e a engenharia de criar máquinas inteligentes, especialmente programas de computador inteligentes. Está relacionada à tarefa semelhante de usar computadores para entender a inteligência humana, mas a IA não precisa se limitar a métodos que sejam biologicamente observáveis \cite[tradução própria]{mccarthy2007}.
\end{citacao}

Poucos anos depois, em 1956, a Conferência de Dartmouth reuniu um grupo de pioneiros --- entre eles Marvin Minsky, Claude Shannon e Herbert Simon --- e estabeleceu o ponto de partida formal para a pesquisa em IA. Esse encontro não só popularizou o termo, mas também incentivou o desenvolvimento de métodos simbólicos e heurísticos para resolver problemas complexos.
